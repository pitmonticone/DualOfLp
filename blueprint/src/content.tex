All vector spaces in this document are over the field $\RR$ of real numbers.
\chapter{Mise en bouche}
\begin{theorem}\label{thm:unique_topology_fd}
\lean{unique_topology_of_finiteDimensional}
\leanok
	The usual topology on $\RR^{n}$ is the only Hausdorff topology that makes it a topological
  vector space.
\end{theorem}
\begin{proof}
\leanok
Let $\TT$ be a Hausdorff topology on $\RR^n$ making it a topological vector space. The identity map $\operatorname{id}\colon \bigl(\RR^n\bigr){\text{can}}\to\bigl(\RR^n\bigr)_{\TT}$, where the first space is endowed with the usual topology and the second with $\TT$ is a continuous (linear) bijection between Hausdorff spaces, and it is therefore a homeomorphism.

Of course, the whole point is to state this in Lean.
\end{proof}

\begin{theorem}\label{thm:discrete_iff_zero}
The only topological vector space over $\RR$ with the discrete topology is the zero space.
\lean{discrete_topology_implies_subsingleton}
\leanok
\end{theorem}
\begin{proof}
\leanok
Let $E$ be a discrete, real, topological vector space: for every $v\in E$, y continuity of the multiplication, the line $\RR\cdot v$ is both connected and discrete, so it is a singleton. This shows $v=0$.
\end{proof}

\chapter{Locally Convex and Bolognese Spaces}
In this chapter, all Banach spaces are supposed to be \emph{non-zero}. Observe that the definition \texttt{BanachSpace} does not exist in Mathlib: before starting off, make sure that you know how to state that a space $V$ is Banach.

\section{Duals of Banach Spaces}
\begin{theorem}\label{thm:banach_exists_mem_dual_eval_ne_zero}
If $V$ is a Banach space and $v\neq 0$, then some $\varphi \in V^*$ satisfies $\varphi(v) \neq 0$.
\lean{nonzero_functional_of_banach}
\leanok
\end{theorem}
\begin{proof}
\leanok
The result is in Mathlib \emph{verbatim}, the problem is to find it.

\emph{Suggestion :} prove first the \textbf{separation result} \ref{thm:banach_exists_mem_dual_eval_ne}.
\end{proof}

\begin{theorem}\label{thm:banach_exists_mem_dual_eval_ne}
Let $V$ be a Banach space and let $v\neq w\neq w$ be two distinct elements of $V$. There is a
$\varphi \in V^*$ such that $\varphi(v) \neq \varphi(w)$.
\lean{separating_dual_of_banach}
\leanok
\end{theorem}
\begin{proof}
\leanok
\uses{thm:bolognese_exists_mem_dual_eval_ne}
See the proof of Theorem~\ref{thm:banach_exists_mem_dual_eval_ne_zero}.
\emph{Extra questions:} Are you sure that completeness is necessary?
\end{proof}

\section{Bolognese Spaces}

\begin{definition}\label{def:bolognese}
\lean{Bolognese}
\leanok
A topological vector space is Bolognese if the convex open
sets are a base for the topology: for every open set $U$ around a point, there is a convex open set
$C$ containing that point such that $C\subseteq U$.
\end{definition}

The above definition is \emph{almost identical} to the definition of \href{https://leanprover-community.github.io/mathlib4_docs/Mathlib/Topology/Algebra/Module/LocallyConvex.html#LocallyConvexSpace}{Locally Convex Space} of Mathlib.

\begin{theorem}\label{thm:bolognese_iff_lc} A topological vector space is Bolognese if and only if
it is Locally Convex.
\lean{bolognese_iff_lc}
\uses{def:bolognese}
\leanok
\end{theorem}
\begin{proof}
\leanok
The proof consists in unfolding the definition of a basis of a topological space, using that when $V$ is a topological vector space (so, addition and scalar multiplications are continuous), there is a basis of the neighborhoods of $0$ made of open, absolutely convex spaces (this is in Mathlib).
\end{proof}

\begin{theorem}\label{thm:Bolognese_of_Banach} Every Banach space is Bolognese.
\lean{bolognese_of_banach}
\uses{def:bolognese}
\leanok
\end{theorem}
\begin{proof}
\leanok
\uses{thm:bolognese_iff_lc}
Since Banach spaces are locally convex, this follows from Theorem~\ref{thm:bolognese_iff_lc}. On the other hand, completeness is probably useless: try to generalise.
\end{proof}

\begin{theorem}\label{thm:Lp_bolognese_of_p_ge_1} For every $1\le p$ the space $L^p([0,1])$
\textup{(}where $[0,1]$ is endowed with the restriction of the Lebesgue measure\textup{)} is Bolognese.
\lean{Lp_bolognese}
\uses{def:bolognese}
\leanok
\end{theorem}
\begin{proof}
\leanok
All the ingredients are in Mathlib: the only ``problem'' is how to state the theorem, namely how to speak about $L^p([0,1])$.
\end{proof}

Finally, we see that Theorem~\ref{thm:banach_exists_mem_dual_eval_ne} was a special case of a more
general result:
\begin{theorem}\label{thm:bolognese_exists_mem_dual_eval_ne}
Let $V$ be a \emph{Hausdorff} Bolognese space and let $v\neq w \in V$ be distinct elements of $V$. There is a
$\varphi \in V^*$ such that $\varphi(v) \neq \varphi(w)$.
\lean{separates_dual_of_bolognese}
\uses{def:bolognese}
\leanok
\end{theorem}
\begin{proof}
\leanok
Once more, the proof is basically a matter of finding the right formulation and right statement in Mathlib. Observe (\emph{i.~e.}: try!) that you can assume T1 instead of Hausdorff, if you so wish.
\end{proof}

\section{Non-Bolognese Spaces}

It comes the question of understanding whether every space is Bolognese and, if not, to produce
examples of non-Bolognese ones.
\begin{theorem}[{Tychonoff (1935), see~\cite{Tyc35}}]\label{thm:lp_not_bolognese}
\lean{lp_not_bolognese}
\uses{def:bolognese}
\leanok
For every $0<p<1$,
the space $\ell^{p}$ of sequences $(x_n)_{n\in\mathbb{N}}$ of real numbers such that
$\sum_{n\geq 0}\lvert x_n\rvert ^p < \infty$ is not Bolognese.
\end{theorem}
\begin{proof}
\leanok
Arguing by contradiction, suppose that the unit ball $\BB{1}\subseteq \ell^p$ contains a convex open $0 \ni U\subseteq \BB{1}$: then $U$ contains a ball $\BB{\varepsilon}$ for some $\varepsilon$.

For each $1\le i\le \infty$, let $x_i=(\varepsilon\delta_{i,j})_{j}$ be the vector whose coordinates are $0$ at all $j\ne i$ and whose $i$-th coordinate is $\varepsilon$: clearly, $x_i\in \BB{\varepsilon}\subseteq \ell^p$. Consider now, for each $N\in\NN$, the vector
\[
y_N=\sum_{i=1}^N \frac{1}{C_N\cdot i^{1/p}}x_i=\bigr(0,\dots,\frac{\varepsilon}{C_Ni^{1/p}}\bigl)
\]
where $C_N=\sum_{i=1}^N i^{1/p}$. Clearly all $y_N$ are (finite) linear combinations of the $x_i$'s, so they all belong to the convex set $U\subseteq \BB{1}$. On the other hand,
\[
\Vert y_N\Vert=\sqrt[p]{\left(\sum_{i=1}^N \frac{\varepsilon}{C_N\cdot i^{1/p}}\right)^p}=
\sqrt[p]{\sum_{i=1}^N \lvert(\frac{\varepsilon^p}{C_N^p\cdot i}\rvert)}=\frac{\varepsilon}{C_N}\sqrt[p]{\sum_{i=1}^N \frac{1}{i}}.
\]
Now, since
\[
\lim_{N\to+\infty}\frac{1}{\sum _{i=1}^N \sqrt[p]{i}}\sqrt[p]{\sum_{i=1}^N \frac{1}{i}}=
\lim_{N\to+\infty}\frac{1}{\sum _{i=1}^N \sqrt[p]{i}}\sqrt[p]{\sum_{i=1}^N \frac{1}{i}}=+\infty
\]
there exists $N$ such that $\Vert y_N\Vert > 1$, contradicting the assumption that $y_N\in\BB{1}$ for all $N$.
\end{proof}

Passing from numerical sequences to measurable functions yields other examples:
\begin{theorem}\label{thm:L_p_not_bolognese}
\lean{L_p_not_bolognese}
\uses{def:bolognese}
\leanok
For every $0<p<1$, the space $L^{p}([0,1])$ of equivalence classes of
measurable functions $f\colon [0,1]\to\RR$ satisfying
$\int_0^1\lvert f(x)\rvert ^{p}\mathrm{d}x < \infty$ \textup{(}with respect to the Lebesgue
measure\textup{)} is not Bolognese.
\end{theorem}
\begin{proof}
\leanok
Combine Theorem~\ref{thm:Lp_dual_zero} and Theorem~\ref{thm:bolognese_exists_mem_dual_eval_ne}.
\uses{thm:Lp_dual_zero, thm:bolognese_exists_mem_dual_eval_ne}
\end{proof}

The above theorem can be proven directly, but we rather deduce it as a corollary of the following one, combined with Theorem~\ref{thm:bolognese_exists_mem_dual_eval_ne}:
\begin{theorem}\label{thm:Lp_dual_zero} For every $0<p<1$ the dual space $L^p([0,1])^\ast$ is equal to $0$.
\lean{Lp_dual_zero}
\uses{def:bolognese}
\leanok
\end{theorem}
\begin{proof}
\leanok
\uses{thm:bolognese_exists_mem_dual_eval_ne}
For a direct proof, see~\cite{Con25}.
\end{proof}


\chapter{Other Non-Bolognese Spaces}
The results in this chapter are all borrowed from~\cite{Con25} and none of them has been formalised during the workshop.
\begin{theorem}\label{thm:Lp_mu_metric} Let $(X,\mu)$ be a measure space. For every $0<p$ the space $L^p(\mu)$ is a metric space with metric
\[
d(f, g)= \begin{cases}
    \left(\int_{X}|f-g|^{p} \mathrm{~d} \mu\right)^{1 / p}, & \text { if } p \geq 1 \\ \int_{X}|f-g|^{p} \mathrm{~d} \mu, & \text { if } 0<p<1
    \end{cases}
\]
\end{theorem}

Can Theorem~\ref{thm:L_p_not_bolognese} be generalised beyond the case $X=[0,1]$? The answer relies on the following
\begin{theorem}\label{thm:Lp_bolognese_iff} For every $0<p<1$, the space $L^p(\mu)$ is Bolognese if and only if $\mu$ assumes finitely many values.
\end{theorem}
As a corollary of the above result, we obtain a new proof of Theorem~\ref{thm:L_p_not_bolognese}:
\begin{corollary}\label{cor:L_p_not_bolognese2} The space $L^{p}([0,1])$ of equivalence classes of measurable functions $f\colon [0,1]\to\RR$ satisfying $\int_0^1\lvert f(x)\rvert ^{p}\mathrm{d}x < \infty$ \textup{(}with respect to the Lebesgue measure\textup{)} is not Bolognese.
\end{corollary}
\begin{proof}
\uses{thm:Lp_bolognese_iff}
\end{proof}
