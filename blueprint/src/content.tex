All vector spaces in this document are over the field $\mathbb{R}$ of real numbers.

\chapter{Mise en bouche}
\begin{theorem}
	The usual topology on $\mathbb{R}^{n}$ is the only Hausdorff topology that makes it a topological vector space.
\end{theorem}

\begin{theorem} The only topological vector space over $\mathbb{R}$ with the discrete topology is the zero space
\end{theorem}

\chapter{Bolognese Spaces}
In this document, a Banach space is non-empty.

\begin{theorem}\label{thm:banach_exists_mem_dual_eval_ne_zero}
If $V$ is Banach and $v\neq 0$, then some $\varphi \in V^*$ satisfies $\varphi(v) \neq 0$.
\end{theorem}

\begin{theorem}\label{thm:banach_exists_mem_dual_eval_ne}
Let $V$ be a Banach space and let $v\neq w\neq w$ be two distinct elements of $V$. There is a $\varphi \in V^*$ such that $\varphi(v) \neq \varphi(w)$.
\end{theorem}

\begin{definition}\label{def:bolognese} A topological vector space is Bolognese if the convex open sets are a base for the topology: for every open set $U$ around a point, there is a convex open set $C$ containing that point such that $C\subseteq U$.
\end{definition}

The above definition is \emph{almost identical} to the definition of \href{https://leanprover-community.github.io/mathlib4_docs/Mathlib/Topology/Algebra/Module/LocallyConvex.html#LocallyConvexSpace}{Locally Convex Space} of Mathlib.

\begin{theorem}\label{thm:bolognese_iff_lc} A topological vector space is Bolognese if and only if it is Locally Convex.
\end{theorem}

\begin{theorem}\label{thm:Bolognese_of_Banach} Every Banach space is Bolognese.
\end{theorem}


\begin{theorem}\label{thm:bolognese_exists_mem_dual_eval_ne}
Let $V$ be a Bolognese space and let $v\neq w \in V$ be distinct elements of $V$. There is a $\varphi \in V^*$ such that $\varphi(v) \neq \varphi(w)$.
\end{theorem}

It comes the question of understanding whether every space is Bolognese and, if not, to produce examples of non-Bolognese ones.
\begin{theorem}\label{thm:l_half_not_bolognese} For every $0<p<1$, the space $\ell^{p}$ of sequences $(x_n)_{n\in\mathbb{N}}$ of real numbers such that $\sum_{n\geq 0}\sqrt{\lvert x_n\rvert}< \infty$ is not Bolognese.
\end{theorem}

More generally,
\begin{theorem}[Tychonoff]\label{thm:lp_not_bolognese} For every $0<p<1$, the space $\ell^{p}$ of sequences $(x_n)_{n\in\mathbb{N}}$ of real numbers such that $\sum_{n\geq 0}\lvert x_n\rvert ^p < \infty$ is not Bolognese.
\end{theorem}

Passing from numerical sequences to measurable functions yields other examples:
\begin{theorem}\label{thm:L_half_not_bolognese} The space $L^{1/2}([0,1])$ of equivalence classes of measurable functions $f\colon [0,1]\to\mathbb{R}$ satisfying $\int_0^1\lvert f(x)\rvert ^{1/2}\mathrm{d}x < \infty$ \textup{(}with respect to the Lebesgue measure\textup{)} is not Bolognese.
\end{theorem}

The following result shows that being Bolognese, or Locally Convex, is really crucial in Theorem~\ref{thm:bolognese_exists_mem_dual_eval_ne}.
\begin{theorem}\label{thm:Lp_dual_zero} For every $0<p<1$ the dual space $L^p([0,1])^\ast$ is equal to $0$.
\end{theorem}
In other words,
\begin{theorem}\label{thm:Lp_cont_funct_zero}  For every $0<p<1$, the only continuous linear
map $\varphi\colon L^{p}([0,1])\to\mathbb{R}$ is the constant function $0$.
\end{theorem}


\chapter{Other Spaces with Trivial Dual}

\begin{theorem}\label{thm:Lp_mu_metric} Let $(X,\mu)$ be a measure space. For every $0<p$ the space $L^p(\mu)$ is a metric space with metric
\[
d(f, g)= \begin{cases}
    \left(\int_{X}|f-g|^{p} \mathrm{~d} \mu\right)^{1 / p}, & \text { if } p \geq 1 \\ \int_{X}|f-g|^{p} \mathrm{~d} \mu, & \text { if } 0<p<1
    \end{cases}
\]
\end{theorem}

\begin{theorem}\label{thm:Lp_bolognese_of_p_ge_1} For every $1\le p$ the space $L^p(\mu)$ is Bolognese.
\end{theorem}

Can Theorem~\ref{thm:L_half_not_bolognese} be generalised beyond the case $X=[0,1]$? The answer relies on the following
\begin{theorem}\label{thm:Lp_bolognese_iff} For every $0<p<1$, the space $L^p(\mu)$ is Bolognese if and only if $\mu$ assumes finitely many values.
\end{theorem}

Let's deduce the following generalisation of Theorem~\ref{thm:L_half_not_bolognese} from Theorem~\ref{thm:Lp_bolognese_iff} rather than generalising the proof of Theorem~\ref{thm:L_half_not_bolognese}:
\begin{theorem}\label{thm:L_p_not_bolognese} The space $L^{p}([0,1])$ of equivalence classes of measurable functions $f\colon [0,1]\to\mathbb{R}$ satisfying $\int_0^1\lvert f(x)\rvert ^{p}\mathrm{d}x < \infty$ \textup{(}with respect to the Lebesgue measure\textup{)} is not Bolognese.
\end{theorem}
