\documentclass[10pt]{article}
\usepackage{amsmath, amsthm, amssymb}

\title{Lp-Spaces for $0 < p < 1$}
\author{Keith Conrad}
\date{}

\theoremstyle{definition}
\newtheorem{definition}{Definition}[section]
\theoremstyle{plain}
\newtheorem{theorem}[definition]{Theorem}
\newtheorem{example}[definition]{Example}

\begin{document}

\maketitle

\section{Introduction}

In a first course in functional analysis, a great deal of time is spent with Banach spaces,
especially the interaction between such spaces and their dual spaces. Banach spaces are a
special type of topological vector space, and there are important topological vector spaces
that do not lie in the Banach category, such as the Schwartz spaces.

The most fundamental theorem about Banach spaces is the Hahn-Banach theorem, which
links the original Banach space with its dual space. What we want to illustrate here is a
wide collection of topological vector spaces where the Hahn-Banach theorem has no obvious
extension because the dual space is zero. The model for a topological vector space with
zero dual space will be $L^p[0,1]$ when $0 < p < 1$. After proving the dual of this space is $\{0\}$,
we’ll see how to make the proof work for other $L^p$-spaces, with $0 < p < 1$. The argument
eventually culminates in a theorem from measure theory (Theorem 4.2).

\section{Banach Spaces and Beyond}

Throughout, vector spaces are real (comments also apply to complex vector spaces).

\begin{definition}
A norm on a vector space $V$ is a function $\|\cdot\| : V \to \mathbb{R}$ satisfying:
\begin{itemize}
\item $\|v\| \ge 0$, with equality iff $v = 0$,
\item $\|v+w\| \le \|v\| + \|w\|$,
\item $\|cv\| = |c|\|v\|$ for all scalars $c$.
\end{itemize}
\end{definition}

Given a norm, define the metric $d(v,w) = \|v-w\|$.

\begin{example}
On $\mathbb{R}^n$ we have the sup-norm $\|x\|_{\sup} = \max_i |x_i|$ and Euclidean norm
$\|x\|_2 = (\sum_i |x_i|^2)^{1/2}$. These induce the same topology.
\end{example}

\begin{example}
On $C[0,1]$ the sup norm and $L^2$-norm differ topologically: a tall narrow spike can be
small in $L^2$ but large in the sup norm.
\end{example}

\begin{definition}
A Banach space is a normed vector space that is complete in its metric.
\end{definition}

\begin{example}
$\mathbb{R}^n$ is Banach under any norm. $C[0,1]$ is Banach under the sup norm but not the
$L^2$ norm.
\end{example}

\begin{definition}
The dual space $V^*$ of a Banach space $V$ is the space of continuous linear maps $V \to \mathbb{R}$.
\end{definition}

A key theorem:

\begin{theorem}
If $V$ is Banach and $v\neq 0$, then some $\varphi \in V^*$ satisfies $\varphi(v) \neq 0$.
\end{theorem}

\begin{example}
Evaluation maps $e_a(f)=f(a)$ are continuous linear functionals on $C[0,1]$ under the sup norm.
\end{example}

\begin{definition}
A topological vector space is a real vector space with a Hausdorff topology such that addition and scalar multiplication are continuous.
\end{definition}

\begin{definition}
A topological vector space is locally convex if convex open neighborhoods of $0$ form a neighborhood base.
\end{definition}

Every Banach space is locally convex, but some natural spaces are not.

\begin{example}
$L^{1/2}[0,1]$ with metric $d(f,g)=\int_0^1 |f-g|^{1/2}$ is not locally convex. No ball around $0$ contains a smaller convex neighborhood.
\end{example}

\begin{example}
For $0<p<1$, $L^p[0,1]$ with metric $d(f,g)=\int_0^1 |f-g|^p$ is also not locally convex.
\end{example}

\begin{theorem}
For $0<p<1$, the dual space $L^p[0,1]^* = \{0\}$.
\end{theorem}

A constructive contradiction argument shows no nonzero continuous linear functional exists.

\section{More Spaces with Dual Space 0}

Let $(X,\mathcal{M},\mu)$ be a measure space.

\begin{definition}
For $p>0$,
\[
L^p(\mu)=\{f\text{ measurable} : \int_X |f|^p\, d\mu < \infty\},
\]
identifying functions equal almost everywhere.
\end{definition}

The metric is
\[
d(f,g) =
\begin{cases}
\left(\int_X |f-g|^p\, d\mu\right)^{1/p}, & p\ge 1,\\[6pt]
\int_X |f-g|^p\, d\mu, & 0<p<1.
\end{cases}
\]

\begin{theorem}
For $0<p<1$, $L^p(\mu)$ is locally convex iff $\mu$ takes only finitely many values.
\end{theorem}

\begin{theorem}
If $\mu$ has an atom of finite measure, then $L^p(\mu)^*\neq \{0\}$.
\end{theorem}

\begin{example}
For $0<p<1$, $\ell^p=L^p(\mathbb{N})$ is not locally convex but has nonzero dual $\ell^\infty$.
\end{example}

\begin{theorem}
If $\mu$ is nonatomic and $0<p<1$, then $L^p(\mu)^*=\{0\}$.
\end{theorem}

\end{document}
